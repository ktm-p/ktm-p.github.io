\documentclass[letterpaper,11pt]{article}

\usepackage{latexsym}
\usepackage[empty]{fullpage}
\usepackage{titlesec}
\usepackage{marvosym}
\usepackage[usenames,dvipsnames]{color}
\usepackage{verbatim}
\usepackage{enumitem}
\usepackage[hidelinks]{hyperref}
\usepackage{fancyhdr}
\usepackage[english]{babel}
\usepackage{tabularx}
\input{glyphtounicode}

\pagestyle{fancy}
\fancyhf{} % Clear all header and footer fields
\fancyfoot{}
\renewcommand{\headrulewidth}{0pt}
\renewcommand{\footrulewidth}{0pt}

% Adjust margins
\addtolength{\oddsidemargin}{-0.5in}
\addtolength{\evensidemargin}{-0.5in}
\addtolength{\textwidth}{1in}
\addtolength{\topmargin}{-.5in}
\addtolength{\textheight}{1.0in}

\urlstyle{same}

\raggedbottom
\raggedright
\setlength{\tabcolsep}{0in}

% Sections formatting
\titleformat{\section}{
	\vspace{-4pt}\scshape\raggedright\large
}{}{0em}{}[\color{black}\titlerule \vspace{-5pt}]

% Ensure that generate PF is machine readable/ATS parsable
\pdfgentounicode=1

%%% CUSTOM COMMANDS %%%

% Bolded Text: Text
\newcommand{\resumeItem}[2]{
	\item\small{
		\textbf{#1}{: #2 \vspace{-2pt}}
	}
}

% Text
\newcommand{\resumeItemTwo}[1]{
	\item\small{
		#1 \vspace{-2pt}}
}

% Text: Text
\newcommand{\resumeItemThree}[2]{
	\item\small{
		#1: #2 \vspace{-2pt}}
}

% Bold Title 					Text
% Itacliciszed Text				Italicized Text
\newcommand{\resumeSubheading}[4]{
	\vspace{-1pt}\item
	\begin{tabular*}{0.97\textwidth}[t]{l@{\extracolsep{\fill}}r}
		\textbf{#1} & #2 \\
		\textit{#3} & \textit{\small #4} \\
	\end{tabular*}\vspace{-5pt}
}

%% Title | Text
\newcommand{\resumeSubheadingTwo}[2]{
	\vspace{-1pt}\item
	\begin{tabular*}{0.97\textwidth}[t]{l@{\extracolsep{\fill}}r}
		\textbf{#1} \rule[-0.4ex]{0.1ex}{1.2em} #2 \\
	\end{tabular*}\vspace{-5pt}
}

% Bold Title 					Text
% Italicized Text				Italicized Text
% Italicized Text
\newcommand{\resumeSubheadingThree}[5]{
	\vspace{-1pt}\item
	\begin{tabular*}{0.97\textwidth}[t]{l@{\extracolsep{\fill}}r}
		\textbf{#1} & #2 \\
		\textit{#3} & \textit{\small #4} \\
		\textit{#5} & \\
	\end{tabular*}\vspace{-5pt}
}

\newcommand{\resumeSubItem}[2]{\resumeItem{#1}{#2}\vspace{-4pt}}

\renewcommand{\labelitemii}{$\circ$}

\newcommand{\resumeSubHeadingListStart}{\begin{itemize}[leftmargin=*]}
	\newcommand{\resumeSubHeadingListEnd}{\end{itemize}}
\newcommand{\resumeItemListStart}{\begin{itemize}}
	\newcommand{\resumeItemListEnd}{\end{itemize}\vspace{-5pt}}

\begin{document}
	
	%%% Heading %%%
	\begin{tabular*}{\textwidth}{l@{\extracolsep{\fill}}r}
		\textbf{\href{https://ktm-p.github.io/}{\Large Michael Pham}} & Email: \href{mailto:ktmpham@berkeley.edu}{ktmpham@berkeley.edu}\\
		\href{https://ktm-p.github.io/}{ktm-p.github.io} & 
		Mobile: \href{tel:+19169680563}{(916)-968-0563} \\
	\end{tabular*}
	
	%%% Education %%%
	\section{Education}
	\resumeSubHeadingListStart
	\resumeSubheading
	{River City High School}{West Sacramento, CA}
	{High School Diploma}{Mar 2019 -- Jun 2022}
	\resumeItemListStart
	\resumeItemThree{\small GPA}{\small 4.00}
	\resumeItemTwo{\small Graduated Salutatorian}
	\resumeItemListEnd
	\resumeSubheadingThree
	{University of California, Berkeley}{Berkeley, CA}
	{B.A. in Computer Science and Mathematics}{Aug 2022 -- Present}
	{Minor in Data Science}
	\resumeItemListStart
	\resumeItemThree{\small GPA}{\small 3.87}
	\resumeItemTwo{\small Member of Upsilon Pi Epsilon Honor Society}
	\resumeItemListEnd
	\resumeSubHeadingListEnd
	
	
	%%% Projects %%%
	\section{Projects}
	\resumeSubHeadingListStart
	
	\resumeSubheadingTwo
	{A Secure File Sharing System}{Golang}
	\resumeItemListStart
	\resumeItemTwo{Designed and implemented a secure file sharing system using cryptographic library functions.}
	\resumeItemTwo{Implemented file creation, appending, sharing, and deletion among multiple users. Users could also sign on from multiple devices and edits would be reflected across all accounts.}
	\resumeItemTwo{Utilized symmetric encryption, HMACs, and digital signatures to ensure security.}
	\resumeItemTwo{Extensively tested implementation, writing over three thousand lines of test code. Utilized fuzzing as well.}
	\resumeItemListEnd
	
	\resumeSubheadingTwo
	{Berkeley Admissions Visualization}{Python, Matplotlib, NumPy, Pandas, Plotly, RegEx, Seaborn}
	\resumeItemListStart
	\resumeItemTwo{Compiled data on Berkeley's Californian public school admissions, and created visualizations for it.}
	%	\resumeItemTwo{Worked with datasets from multiple different sources with different formatting.}
	\resumeItemTwo{Filtered, regularized, and merged data from various sources with Pandas and RegEx.}
	\resumeItemTwo{Visualized data using scattermaps, choropleth maps, and other charts using Matplotlib, Seaborn, and Plotly.}
	\resumeItemListEnd
	
	\resumeSubheadingTwo{Machine Learning}{Python, PyTorch}
	\resumeItemListStart
	\resumeItemTwo{Used PyTorch on a variety of machine learning problems. Approximated a sinusoidal curve. Additionally, implemented language detection and handwriting recognition.}
	\resumeItemTwo{Utilized a two-layer Recurrent Neural Network for language recognition of words of differing lengths. Achieved an accuracy of over 80\%.}
	\resumeItemTwo{Implemented a two-layer Linear Neural Network with ReLU activation function and Cross-Entropy Loss for handwriting recognition. Filtered the data using convolution, and then flattened it to enhance model performance. Achieved an accuracy of over 98\%.}
	\resumeItemListEnd
	
	\resumeSubheadingTwo{Reinforcement Learning}{Python, PyTorch}
	\resumeItemListStart
	\resumeItemTwo{Utilized Reinforcement Learning to train Pac-Man agent to win. Achieved a win rate of over 90\%.}
	\resumeItemTwo{Implemented value iteration, Q-Learning, Approximate Q-Leaning, and Deep Q-Learning using PyTorch.}
	\resumeItemTwo{Used Multi-Layered Linear Neural Network with ReLU activation and Mean Square Error Loss in Deep Q-Learning. Fine-tuned hyperparameters such as learning rate, hidden layer sizes, and number of training episodes.}
	\resumeItemListEnd
	
	\resumeSubheadingTwo
	{Spam Classifier}{Python, Matplotlib, NumPy, Pandas, RegEx, scikit-learn, Seaborn}
	\resumeItemListStart
	\resumeItemTwo{Created a spam email filter using a Logistic Regression model. Achieved an accuracy of 99.2\% on given test data.}
	\resumeItemTwo{Cleaned and visualized data using Pandas, RegEx, Matplotlib, and Seaborn.}
	\resumeItemTwo{Fine-tuned hyperparameters by cross-validation with GridSearchCV.}
	\resumeItemListEnd
	
	\resumeSubHeadingListEnd
	
	%%% Skills %%%
	\section{Technical Skills}
	\resumeSubHeadingListStart
	\resumeSubItem{Programming Languages}
	{C, CSS, Golang, HTML, Java, Javascript, MATLAB, Python, R, RISC-V, Scheme, SQL}
	\resumeSubItem{Frameworks/Libraries}{Matplotlib, Numpy, OpenMP, OpenMPI, Pandas, Plotly, Processing, PyTorch, scikit-learn, Seaborn, TensorFlow}
	\resumeSubItem{Tools}{Docker, gdb, git, Logism, LaTeX, Valgrind}
	\resumeSubItem{Mathematics}{Abstract Algebra, Discrete Mathematics, Linear Algebra, Linear Programming, Logic, Numerical Analysis, Real Analysis}
	\resumeSubHeadingListEnd
\end{document}